This section gives example input files and instructions for a typical C\+I\+N\+EB calculation run through C\+DE.

In this case, we are going to run a C\+I\+N\+EB optimization of a reaction path representing dissociation of molecular hydrogen (H2) from formaldehyde (H2\+CO). We will using the semi-\/empirical P\+M3 method {\itshape via} the {\itshape ab initio} code {\itshape O\+R\+CA}, to calculate potential energies and forces.

{\bfseries The actual files to run this example can be found in the $\ast$$\sim$/cde/examples/\+Tutorial\+\_\+1 directory.}

In this directory, you will find several input files\+: \begin{DoxyVerb}  input
  initial_path.xyz
  orca.head
  orca.min
\end{DoxyVerb}


All of these input filenames are arbitrary; we could have called them alice, bob, clive and ralph and the program would still read them in correctly. Note that the extensions have no effect for input files read in by C\+DE, but they can be used to remind you what the different input files are.

Let\textquotesingle{}s have a look at each input file individually\+:

\subsubsection*{input}

The {\itshape input} file looks like this\+: \begin{DoxyVerb}# This input file demonstrates a CINEB refinement.
# We use the quickmin method here, and orca
# for PES calculations.

nimage 8
temperature 10
calctype optpath
startfile start.xyz
endfile end.xyz
pathfile path.xyz
ranseed  5
startfrompath  .FALSE.
pathinit  linear

# path optimization

pathoptmethod cineb
nebmethod quickmin
nebiter 250
vsthresh 1d-2
cithresh 1d-3
nebspring 0.05
nebconv 2d-3
nebstep  10.0
neboutfreq 5
stripinactive .FALSE.
optendsbefore .TRUE.
optendsduring .false.
nebrestrend .FALSE.
alignedatoms
1 2 3

# constraints

dofconstraints 0
atomconstraints 0

# PES input

pesfull .true.
pestype  orca
pesfile   orca.head
pesopttype  orca
pesoptfile orca.min
pesexecutable orca
pesoptexecutable orca
\end{DoxyVerb}


The input file in this case only contains those options which are relevant to the C\+I\+N\+EB simulation; other input directives have been removed and will be ignored internally in C\+DE.

The first input block (beginning with the comment line \char`\"{}\# control parameters\char`\"{}) identifies the following input parameters\+:
\begin{DoxyItemize}
\item {\bfseries nimage}\+: Number of images (in total) in the reaction-\/path which is going to be optimized. In this case, we have 8; note that the number of images indicated here must be the same as in the input pathfile if {\itshape startfrompath = .T\+R\+UE.} (see below).
\item {\bfseries temperature}\+: We\textquotesingle{}re going to use the {\itshape quickmin} optimization algorithm to refine our path (see below). In {\itshape quickmin}, each atom in each string is given a velocity to move it towards the M\+EP; these initial velocities are drawn from a Boltzmann distribution at the input temperature (in Kelvin) indicated here. For {\itshape quickmin} optimization, this temperature should be relatively low.
\item {\bfseries calctype}\+: Here, we\textquotesingle{}re using {\itshape optpath} to indicate that we want to do a path optimization calculation.
\item {\bfseries pathfile}\+: In nudged elastic band path refinement calculations, we often have a good initial guess of the reaction path we\textquotesingle{}d like to refine. The {\itshape pathfile} variable defines the input file where our initial reaction-\/path guess is stored. In this case, we\textquotesingle{}re saying that our initial reaction-\/path is in {\itshape path.\+xyz}; this is an {\itshape xyz} format file (with coordinates given in Angstroms) which M\+U\+ST contain the same number of snapshots as the number {\itshape nimage} defined above.
\end{DoxyItemize}

{\bfseries I\+M\+P\+O\+R\+T\+A\+NT\+: In this example, although we\textquotesingle{}re showing how to define a starting {\itshape pathfile}, we\textquotesingle{}re note actually going to use it! Instead, we\textquotesingle{}ve set {\itshape startfrompath .F\+A\+L\+SE.}, which means that the initial path is instead going to be generated by linear interpolation of coordinates between the {\itshape startfile} and {\itshape endfile}.}


\begin{DoxyItemize}
\item {\bfseries ranseed}\+: This is an integer random number seed, used to generate random initial velocities drawn from the Boltzmann distribution for {\itshape quickmin} (see above)
\item {\bfseries startfile}\+: This is an xyz file containing the reactant structure for our target system.
\item {\bfseries endfile}\+: This is an xyz file containing the product structure for out target system. {\bfseries Note that the {\itshape startfile} and {\itshape endfile} M\+U\+ST have the same number of atoms in the same order!}
\item {\bfseries startfrompath}\+: Here, by inputting \char`\"{}.\+F\+A\+L\+S\+E.\char`\"{}, we\textquotesingle{}re indicating that C\+DE should instead start by generating its own initial path by linear interpolation between {\itshape startfile} and {\itshape endfile}.
\end{DoxyItemize}

The second input block (beginning with \char`\"{}\# path optimization\char`\"{}) defines parameters which control the N\+EB path refinement calculation\+:


\begin{DoxyItemize}
\item {\bfseries pathoptmethod}\+: We\textquotesingle{}re setting this to {\itshape cineb}, to indicate that we\textquotesingle{}d like to run a C\+I\+N\+EB calculation.
\item {\bfseries nebmethod}\+: This defines the optimization method we\textquotesingle{}re using to find the minimum-\/energy path. Options are {\itshape quickmin}, which uses a velocity Verlet-\/based method, or {\itshape steepest} for a simple steepest-\/descent method. The {\itshape quickmin} method is preferred -\/ it\textquotesingle{}s usually much faster to converge.
\item {\bfseries nebiter}\+: This defines the mamximum number of C\+I\+N\+EB iterations to perform before stopping. Note that if the calculation reaches a force convergence value lower than {\itshape nebconv} (see below), it stops anyway.
\item {\bfseries cithresh}\+: This is the force threshold at which the climbing-\/image variant of N\+EB (see references for details) is activated. In this case, once the current reaction-\/path forces are less than 0.\+008 Eh/a0, climbing-\/image is activated.
\item {\bfseries nebconv}\+: Target force convergence threshold for C\+I\+N\+EB calculation.
\item {\bfseries nebspring}\+: C\+I\+N\+EB sprin parameter (in atomic units, Eh/a0$\ast$$\ast$2).
\item {\bfseries nebstep}\+: Step-\/length parameter for optimization. In the case of {\itshape quickmin}, this is an effective velocity Verlet time-\/step taken at each C\+I\+N\+EB iteration (in atomic units, so a value of $\sim$41 au corresponds to about a 1 fs time-\/step). In the case of steepest-\/descent ({\itshape nebmethod steepest}), the {\itshape nebstep} parameter indicates the fraction of the current force used to update the atomic coordinates at each step.
\item {\bfseries neboutfreq}\+: Frequency with which the N\+EB calculation will output the current reaction path and energy profile. The energy profiles are input to the file ending with $\ast$.nebprofile$\ast$ and the paths are output to {\itshape input\+\_\+\+Y\+Y\+Y\+Y.\+xyz}, where {\itshape Y\+Y\+YY} is the current N\+EB iteration.
\item {\bfseries stripinactive}\+: If set toe $\ast$.T\+R\+UE.$\ast$, then any molecules which are simply acting as \char`\"{}spectators\char`\"{} to the reaction (e.\+g. they are far away from the reaction sites and do not participate) are \char`\"{}stripped\char`\"{} (removed) from the reaction path B\+E\+F\+O\+RE the N\+EB calculation. This option can make calculations a bit faster!
\item {\bfseries optendsbefore}\+: If set to $\ast$.T\+R\+UE.$\ast$, then C\+DE first performs a geometry optimization of the two end-\/point structures {\itshape before} procedding with C\+I\+N\+EB.
\item {\bfseries optendsduring}\+: If set to $\ast$.T\+R\+UE.$\ast$, the two end-\/points are optimized under the action of the P\+ES {\itshape during} the N\+EB optimization. Here, the end-\/points only feel the force due to the P\+ES, and do not feel the N\+EB springs.
\item {\bfseries nebrestrend}\+: If {\itshape optendsduring = .T\+R\+UE.}, this options indicates whether or not to additionally include the graph-\/restraining potential into the forces felt by the two end-\/points during N\+EB optimization. This option can prevent the end-\/point molecules changing the connectivity during N\+EB optimization. Note that this option is only important if {\itshape optendsduring = .T\+R\+UE.}.
\item {\bfseries alignedatoms}\+: As described in the {\itshape annotated input file description} (\mbox{\hyperlink{_annotated}{Annotated input file description}}), the {\itshape alignedatoms} directive tells C\+DE which triple of atoms should serve as coordinates to align and orient the N\+EB string in space, to avoid overall translation and rotation of the system during N\+EB refinement.
\end{DoxyItemize}

The next input block (starting with \char`\"{}\# constraints\char`\"{}) indicate any constraints we\textquotesingle{}d like to impose on each image in the reaction-\/path during N\+EB optimization.


\begin{DoxyItemize}
\item {\bfseries atomconstraints}\+: In this example, we\textquotesingle{}ve indicated that there are no atom constraints (beyond the {\itshape alignedatoms} directive noted above).
\item {\bfseries dofconstraints}\+: Again, in this case, we don\textquotesingle{}t have any D\+OF constraints other than those noted above.
\end{DoxyItemize}

The final block (starting with \char`\"{}\# P\+E\+S input\char`\"{}) defines options to enable potential energy evaluations and geometry optimization during the N\+EB calculation. These input parameters are discussed elsewhere, such as in {\itshape Annotated input file description} (\mbox{\hyperlink{_annotated}{Annotated input file description}}).

\subsubsection*{start.\+xyz and end.\+xyz}

Here, the file {\itshape start.\+xyz} and {\itshape end.\+xyz} files contain the reactant and product structures for the initial reaction-\/path. These are xyz files, as described in {\itshape I/O formats} (\mbox{\hyperlink{formats}{I/O structure formats}}).

\subsubsection*{orca.\+head}

In the current calculation, we\textquotesingle{}re going to use the {\itshape ab initio} code {\itshape O\+R\+CA} to perform potential energy and force evaluations.

Again, note that we could have called this file anything we wanted to -\/ it doesn\textquotesingle{}t have to be called {\itshape orca.\+head}. However, we must make sure that this is the same filename as defined in the {\itshape pesfil} variable in the input file above.

The {\itshape orca.\+head} file looks like this. \begin{DoxyVerb}! PM3 ENGRAD
* xyz 0 1
XXX
*
\end{DoxyVerb}


Note that this file is exactly the same format as a usual {\itshape O\+R\+CA} input file\+:


\begin{DoxyItemize}
\item The first line tells {\itshape O\+R\+CA} to run an {\itshape energy and gradient} evaluation using the {\itshape P\+M3} semi-\/empirical method.
\item The second line begins the geometry specification; here, we\textquotesingle{}re indicating that the geometry will be specified as {\itshape xyz} Cartesian coordinates (in Angstroms), the total charge on the system is {\itshape 0}, and the total spin multiplicity is {\itshape 1} (i.\+e. closed-\/shell species).
\item In a normal {\itshape O\+R\+CA} file, we would then input the geometry on the following lines. For example, an {\itshape O\+R\+CA} input file would look like this\+: \begin{DoxyVerb}  ! PM3 ENGRAD
  * xyz 0 1
  C 0.1500  0.2166 0.007
  O 1.56  1.679  5.7098
  *
\end{DoxyVerb}

\item However, in an input file for C\+DE, the coordinate block is simply replaced by the string {\itshape X\+XX}. This indicates that this is the point at which {\itshape C\+DE} should insert coordinates for molecules it would like to run through {\itshape O\+R\+CA}.
\item Finally, there is a \char`\"{}$\ast$\char`\"{} symbol on the last line to indicate that this is the end of the {\itshape O\+R\+CA} input file.
\end{DoxyItemize}

\subsubsection*{orca.\+min}

Like the {\itshape orca.\+head} file, the file defined as {\itshape pesoptfile} should also be present in the directory.

This file indicates how to run a geometry optimization calculation; note that this calculation is only used if {\itshape optendsbefore = .T\+R\+UE.}.

As in the case of the files above, the {\itshape pesoptfile} can be called anything you\textquotesingle{}d like!

The format of the {\itshape pesoptfile} (here called {\itshape orca.\+min}) is the same as an {\itshape O\+R\+CA} file for a geometry optimization, but again with the condition that the coordinate input block is replaced by {\itshape X\+XX}. An example is\+: \begin{DoxyVerb}! PM3 OPT TightSCF
* xyz 0 1
XXX
*
\end{DoxyVerb}


Note that everything in the file above is the normal {\itshape O\+R\+CA} format; the only difference is the {\itshape X\+XX} line required by C\+DE.

\subsubsection*{A note on executables for external programs}

Note that our input file above contains two variables defined as\+: \begin{DoxyVerb}  pesexecutable orca
  pesoptexecutable orca
\end{DoxyVerb}


The {\itshape pesexecutable} and {\itshape pesoptexecutable} variables indicate which code to run to calculate energies and forces; in this case, we\textquotesingle{}re using {\itshape O\+R\+CA}.

Note that, in each case, the input command is run directly. For example, based on the above, C\+DE will run the following command when an {\itshape O\+R\+CA} calculation is required\+: \begin{DoxyVerb}orca temp.in
\end{DoxyVerb}


where {\itshape temp.\+in} is the name of a temporary input file which is auto-\/generated by C\+DE during N\+EB calculations.

{\bfseries If there is no executable called \char`\"{}orca\char`\"{}, the C\+DE calculation will fail!}

To make sure everything runs smoothly, there are two options\+:

(1) You can set up an alias so that, when C\+DE executes the command {\itshape orca} given as {\itshape pesexecutable} and {\itshape pesoptexecutable}, everything runs as expected. For example, lets say you have {\itshape O\+R\+CA} installed in $\ast$$\sim$/programs/stuff/orca/bin/$\ast$. If you include an alias in your $\ast$.bashrc$\ast$ (or similar config file if you\textquotesingle{}re on a different system) which reads \begin{DoxyVerb}alias orca='~/programs/stuff/orca/bin/orca'
\end{DoxyVerb}


then running the command {\itshape orca} will then execute $\ast$$\sim$/programs/stuff/orca/bin/orca$\ast$, which should indeed correspond to an executable binary.

(2) As an alternative, you can also give the full pathname to the deisred executable directly in the input file, lik this\+: \begin{DoxyVerb}  pesexecutable ~/programs/stuff/orca/bin/orca
  pesoptexecutable ~/programs/stuff/orca/bin/orca
\end{DoxyVerb}


\subsubsection*{Running the calculation}

With these input files, we are now able to run the C\+I\+N\+EB optimization. To do so, go into the $\ast$$\sim$/cde/examples/cineb/$\ast$ directory and type\+: \begin{DoxyVerb}cde.x input
\end{DoxyVerb}


The above assumes that you have already made sure that C\+DE can be run by simply typing {\itshape cde.\+x}. See the \mbox{\hyperlink{setup}{setup section}} for more details.

As the calculation runs, you will find lots of output files generated in the run directory. Many of these files will be useless outputs generated by {\itshape O\+R\+CA}, typically ocntaining wavefunction and integration grid information.

The interesting output files are as follows\+:


\begin{DoxyItemize}
\item {\bfseries input.\+nebconv}\+: This file shows the total norm of the forces, averaged over the images. These values are given for calculations performed with and without the N\+EB spring terms. This file can be monitored during the calculation to find out how close to convergence the N\+EB calculation is. This file is written after every N\+EB iteration.
\item {\bfseries input.\+nebprofile}\+: This file contains the energy profile along the reaction-\/string at the different iterations of the N\+EB calculation; both absolute energy and relative energy are given. This file is output every $\ast$$\ast$neboutfreq$\ast$ steps of the N\+EB optimization.
\item {\bfseries input\+\_\+\+Y\+Y\+Y\+Y.\+xyz}\+: These files contain the refined path at N\+EB iteration {\itshape Y\+Y\+YY}. These files are output every {\itshape neboutfreq} steps of the N\+EB optimization, and are in standard X\+YZ format. They can be visualized in {\itshape V\+MD} or {\itshape Avogadro}.
\item {\bfseries input.\+energy-\/neb-\/start}\+: This contains the energy profile of the reaction string at the start of the calculation.
\item {\bfseries input.\+energy-\/neb-\/end}\+: This contains the energy profile of the reaction string at the end of the calculation.
\end{DoxyItemize}

In addition to the above ouptut files, the $\ast$.log file produced by all C\+DE calculations will also contain some useful information about the N\+EB run.

\subsubsection*{Next steps}

Possible further steps include\+:


\begin{DoxyItemize}
\item Performing geometry optimization, followed by frequency calculations, for the reaction-\/path end-\/points in order to calculate the free energies using the harmonic oscillator/rigid rotor approximation.
\item Finding the transition state for the reaction, using the image nearest to the top of the reaction-\/barrier obtained by N\+EB.
\item Using all of the above information to calculte the transition-\/state theory rate constant. 
\end{DoxyItemize}